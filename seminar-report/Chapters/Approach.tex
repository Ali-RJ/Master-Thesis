\chapter{شرح روش پیشنهادی}
\label{chap:approach}

در این فصل، قصد داریم روشی برای بهینه‌سازی مسیر انجام 
\gls{DriveTest}
ارائه دهیم که با استفاده از آن، دیگر نیازی به بررسی تمامی موقعیت‌ها و نقاط جغرافیایی یک ناحیه از نقشه نیست و می‌توان با پیمایش یک مسیر کوتاه‌تر، به جمع‌آوری داده‌هایی که نشان‌دهنده‌ی وضعیت سیگنال در آن ناحیه هستند، پرداخت. 

در ابتدا، مدل سامانه و فرضیات مساله در 
\autoref{sec:systemmodel}
مورد بررسی قرار می‌گیرد و پس از آن، در 
\autoref{sec:proposedapproach}،
روش پیشنهادی در چهار گام تشریح می‌شود. این رویکرد به گونه‌ای طراحی شده است تا عملگرهای شبکه‌های تلفن همراه بتوانند با بهبود فرآیند جمع‌آوری داده‌ها، کارایی و بهره‌وری عملیات \gls{DriveTest} را افزایش دهند و به‌ طور همزمان هزینه‌ها و زمان مورد نیاز برای انجام این عملیات‌ها را کاهش دهند.

\section{مدل سامانه و فرضیات}
\label{sec:systemmodel}


\begin{table}
	\centering
	\renewcommand{\arraystretch}{2}
	\caption{فهرست نمادها}
	\begin{tabular}{cp{13cm}}
		\toprule
		\textbf{نماد} &  \textbf{توضیحات} \\
		\midrule
		$W\times H$ &
		ابعاد بخش مستطیلی از نقشه \\
		$K$ &
		تعداد ناحیه‌های مستطیلی کوچک \\
		$N$ &
		تعداد کل نقاط بحرانی‌ انتخاب شده برای \gls{DriveTest} \\
		$M$ &
		حداکثر تعداد نقاط بحرانی‌‌ای انتخاب شده در هر ناحیه \\
		$\mathbf{B}^i$ &
		مجموعه‌ی ایستگاه‌های پایه ناحیه $i$-ام \\
		$\mathbf{P}^i$ &
		مجموعه‌ی نقاط بحرانی ناحیه $i$-ام \\
		$\mathbb{C}(i, d_i)$ &
		دایره‌ای به مرکز نقطه‌ی مرکزی ناحیه‌ی $i$-ام و شعاع $d_i$ به اندازه‌ی قطر این ناحیه‌ی مستطیلی \\
		$\mathbf{O}^i$ &
		اجتماع ایستگاه‌های پایه و نقاط بحرانی ناحیه‌ی $i$-ام در دایره‌ی $\mathbb{C}(i, 0.7d_i)$
		\\
		$(\text{lat},\text{lon})$ &
		مختصات نقطه‌ای در دستگاه مختصات جغرافیایی \\
		$\text{lat}_{\min}^i$ &
		حداقل عرض جغرافیایی نقاط ناحیه‌ی $i$-ام \\
		$\text{lon}_{\max}^i$ &
		حداکثر طول جغرافیایی نقاط ناحیه‌ی $i$-ام \\
		\bottomrule 
	\end{tabular}
	\label{tab:symbols}
\end{table}


\section{تشریح روش پیشنهادی}
\label{sec:proposedapproach}
\begin{nttheorem}
	\label{theorem:betweenmeasure}
	در صورتی‌که نسبت جابه‌جایی بین دو اندازه‌گیری متوالی با فاصله یکی از آن‌ها به اندازه کافی کوچک باشد، می‌توان نتیجه گرفت که 
	$d_{i+1} \approx d_i$.
\end{nttheorem}
\begin{proof}
	مثلث مشخص شده در 
%	\autoref{fig:estimateShadowingNoise}
	را یک بار دیگر در نظر بگیرید. از روابط مثلثاتی می‌دانیم که 
	\begin{equation}
		l_{i,i+1}^2 = d_{i+1}^2+d_i^2 - 2d_id_{i+1}\cos\alpha,
		\label{eq:liiid}
	\end{equation}
	که در آن $l_{i,i+1}$ بیانگر میزان جابه‌جایی بین دو اندازه‌گیری متوالی است.
	\\*[1mm]
\end{proof}

اکنون لم زیر را بدین‌منظور در نظر بگیرد.
\begin{lemma}
	\gls{RandomVariable} $\mathrm{P}^d$
	از
	\gls{GaussianDistribution}
	با 
	\gls{Average}  صفر و \gls{StandardDeviation} $\sqrt{2}\sigma$
	پیروی می‌کند. 
\end{lemma}
\begin{lemmaproof}
	می‌دانیم که 
	$n_i\sim\mathcal{N}(0,\sigma)$. 
	و برای نویز

	
	
\end{lemmaproof}


نمایی از الگوریتم پیشنهادی به صورت سودوکد در 
\autoref{alg:tsp}
نشان  داده شده است. 
\begin{algorithm}[t]
	\caption{حل تقریبی مسئله \lr{TSP}}
	\label{alg:tsp}
	\begin{latin}
		\raggedright
		\textbf{Input}: $N$ Critical points, $K$ partitions.\\
		\vspace{2mm}\textbf{Output}: Approximate TSP solution.
		\begin{algorithmic}[1]
			\STATE\vspace{2mm} Initialize $curPart \gets$ bottom-left partition
			\STATE\vspace{2mm} Initialize $curPoint \gets$ a random point in $curPart$
			\STATE\vspace{2mm} Mark $curPoint$ as visited
			\vspace{2mm}\REPEAT
			\vspace{2mm}\WHILE{unvisited points in $curPart$}
			\STATE\vspace{2mm} Find nearest point $nextPoint$ to $curPoint$
			\STATE\vspace{2mm} Mark $nextPoint$ as visited
			\STATE\vspace{2mm} $curPoint \gets nextPoint$
			\vspace{2mm}\ENDWHILE
			\vspace{2mm}\IF{unvisited partitions remain}
			\STATE\vspace{2mm} Move to next partition (spiral/row-by-row)
			\STATE\vspace{2mm} $curPart \gets$ next partition
			\STATE\vspace{2mm} Find nearest point in $curPart$ to $curPoint$
			\STATE\vspace{2mm} $curPoint \gets$ this nearest point
			\STATE\vspace{2mm} Mark $curPoint$ as visited
			\vspace{2mm}\ENDIF
			\vspace{2mm}\UNTIL{all points visited}\vspace{2mm}
		\end{algorithmic}
	\end{latin}
\end{algorithm}


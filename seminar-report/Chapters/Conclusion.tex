\chapter{نتیجه‌گیری و کا‌ر‌های آینده}
\label{chap:conclusion}


در این فصل در ابتدا در 
\autoref{sec:conclusion}،
به نتیجه‌گیری مروری برآنچه گفته شد پرداخته شده و سپس در
\autoref{sec:futurework}،
مسائل باز این حوزه پرداخته شده و در آخر پیشنهاداتی برای کارهای آینده، ارائه خواهد شد. 

\section{نتیجه‌گیری}
\label{sec:conclusion}
این مجموعه از پژوهش‌ها به وضوح نشان می‌دهد که چگونه نظام‌فکری 
\gls{Model-Driven Deep Learning}،
به‌ویژه از طریق 
\gls{Algorithm Deep Unfolding}،
 توانسته است بر چالش‌های محاسباتی سنگین در سامانه‌های مخابراتی نسل بعدی غلبه کند و در عین حال به عملکردی نزدیک به بهینه‌سازی‌های مرسوم دست یابد. نتیجه‌گیری جامع از بررسی این مقالات تأکید می‌کند که ادغام دانش علمی و الگوریتمی با قدرت شبکه‌های عصبی قابل آموزش، مسیری کارآمد و تفسیرپذیر برای حل مسائل حیاتی پردازش سیگنال و مدیریت منابع در شبکه‌های نسل ششم فراهم می‌آورد.

در هسته این پیشرفت‌ها، 
\gls{Algorithm Deep Unfolding}
قرار دارد که با حفظ ساختار اساسی الگوریتم‌های تکرارشونده مرسوم، مانند الگوریتم‌های گرادیان نزولی،
\gls{ADMM}
 و الگوریتم‌های پیام‌گذرانی، این فرآیندها را به شبکه‌های عصبی قابل آموزش تبدیل می‌کند. مزیت اصلی این رویکرد در تفسیرپذیری بالا نهفته است، زیرا ساختار شبکه به طور مستقیم منطق بهینه‌سازی الگوریتم زیربنایی را دنبال می‌کند و این شبکه‌ها نیازمندی به حجم عظیم داده‌های آموزشی را کاهش داده و در عین حال سرعت استنتاج بالایی را برای عملکرد بلادرنگ تضمین می‌کنند.

در حوزه تخصیص توان و مدیریت منابع، پژوهش‌ها نشان می‌دهند که 
\gls{Deep Unfolding}
 توانسته است کارایی محاسباتی را به شکل چشمگیری افزایش دهد. به عنوان مثال، در مسئله پیچیده کنترل توان با بهره‌وری انرژی در شبکه‌های چند سلولی، مدل‌های
\gls{Deep Unfolding}
  پیشنهادی، مانند 
\gls{FUM}
  و 
\gls{MASUM}،
   به دقت بالایی دست یافتند (
\gls{FUM}
    تا 
\lr{99.32\%} 
   دقت) و سرعت استنتاج را در حد میلی‌ثانیه حفظ کردند. در مقابل، مدل‌های کاملاً جعبه سیاه با بزرگ‌تر شدن اندازه مسئله، عملکرد ضعیف‌تری (حدود
\lr{6.8\%}
 کمتر از الگوریتم‌های مرسوم) از خود نشان دادند، که برتری مدل‌های مدل‌محور را در مقیاس‌پذیری تأیید می‌کند. همچنین، با استفاده از 
\gls{Graph Neural Network}
 در گسترش الگوریتم 
\gls{WMMSE}
  (موسوم به 
\gls{UWMMSE})،
 مشخص شده است که می‌توان به طور مؤثر از خاصیت
\gls{Permutation Equivariance}
    شبکه‌های بی‌سیم استفاده کرد و به عملکردی نزدیک به بهینه در بیشینه‌سازی نرخ مجموع دست یافت، در حالی که امکان پیاده‌سازی غیرمتمرکز و مقیاس‌پذیری به شبکه‌هایی با اندازه‌های دیده نشده فراهم می‌شود.

در بحث تخمین کانال، به‌خصوص در مواجهه با سامانه‌های نوظهوری مانند 
\gls{mmWave}
 کمک‌گرفته از سطوح بازتابنده هوشمند، 
\gls{Deep Unfolding}
 راهگشا بوده است. با توسعه شبکه‌هایی مانند
\gls{RCTS-LAMP}
 و 
\gls{RCTS-LAMP-MMV}
 که 
\gls{Learned Approximate Message Passing}
  را گسترش می‌دهد، محققان توانسته‌اند مسئله دشوار تخمین کانال آبشاری و کانال مستقیم را که به عنوان یک مسئله
\gls{Sparse Recovery}
  در نظر گرفته می‌شود، با موفقیت حل کنند. ارزیابی عملکرد این شبکه‌ها نشان می‌دهد که ترکیب مزایای حسگری فشرده 
\gls{Compressed Sensing}
   با قدرت 
\gls{Deep Learning}،
 امکان تخمین کانال با دقت بالا را حتی در سناریوهایی که کانال مستقیم مسدود شده است، فراهم می‌سازد.
در زمینه سنجش و ارتباطات یکپارچه، که نیازمند طراحی دو منظوره شکل موج‌ها است،
\gls{Deep Unfolding}
 ابزاری حیاتی برای حل مسائل بهینه‌سازی مقید و غیرمحدب پیچیده است. 

طرح‌های مبتنی بر 
\gls{Deep Unfolding}،
  مانند 
\lr{ADMM-NET}،
 برای طراحی شکل موج‌های مدول ثابت 
\gls{Constant Modulus Waveform}
  و شکل‌دهی پرتو غیرفعال مشترک در سامانه‌های 
\gls{ISAC}
   کمک‌گرفته از 
\gls{RIS}
    معرفی شده‌اند. مهم‌ترین نتیجه در این حوزه، کاهش قابل توجه زمان اجرا در مقایسه با الگوریتم‌های بهینه‌سازی مرسوم (مانند
\lr{Branch-and-Bound}
   یا روش‌های مبتنی بر 
\lr{ADMM}
    تکرارشونده)، به‌ویژه در طراحی شکل موج‌هایی با مدول ثابت است، که نشان‌دهنده کارایی بالای 
\gls{Deep Unfolding}
    برای پاسخ‌های سریع در سامانه‌های دو منظوره ‌است.
علاوه بر این، روش 
\gls{Deep Unfolding}
 فراتر از لایه فیزیکی مرسوم گسترش یافته و در رمزگشایی کدهای تصحیح خطا نیز به کار رفته است. در مواجهه با کانال‌های دارای خطا‌های درجی و حذفی، که مدل‌سازی و آشکارسازی در آن‌ها بسیار دشوار است، روش 
\lr{FBNet}
  (گسترش الگوریتم 
\gls{Forward-Backward})
  معرفی شد که نتایج شبیه‌سازی نشان می‌دهند که این روش مدل‌محور می‌تواند با پیچیدگی محاسباتی کم، به عملکردی نزدیک به حداکثر احتمال پسین دست یابد. این حوزه همچنین اهمیت رویکردهای کاملاً داده‌محور را نیز تأیید می‌کند؛ جایی که 
\lr{FBGRU}
   (مبتنی بر
\lr{Bi-GRU})،
    اگرچه تفسیرپذیری کمتری دارد، اما در شرایط 
\gls{Channel State Information Uncertainty}،
   برتری قابل توجهی در استحکام نسبت به همتایان مدل‌محور خود از خود نشان می‌دهد، که تأکیدی بر نیاز به رویکردهای تطبیقی برای مدل‌های کانال ناشناخته است.

در مجموع، نتایج این پژوهش‌ها به‌وضوح برتری نظام‌فکری دانش‌محور، به‌ویژه از طریق 
\gls{Deep Unfolding}،
 را در مقایسه با روش‌های کاملاً داده‌محور و الگوریتم‌های مرسوم، در زمینه‌های حساس به زمان و مقیاس‌پذیر شبکه‌های بی‌سیم نسل ششم اثبات می‌کنند و نشان می‌دهند که برای دستیابی به عملکرد نزدیک به بهینه، تفسیرپذیر و سریع، ادغام هوشمندانه دانش حوزه در طراحی شبکه عصبی ضروری است.
\section{کارهای آینده}
\label{sec:futurework}
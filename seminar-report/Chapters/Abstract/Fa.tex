\clearpage
\section*{چکیده}
شبکه‌های سیار نسل آینده از جمله نسل ششم، با رشد تصاعدی تقاضا برای منابع ناشی از افزایش دستگاه‌های متصل، مواجه هستند. نیازهای کلیدی نسل ششم شامل نرخ داده بالا (یک ترابیت بر ثانیه)، بازده طیفی بالا، اتصال انبوه مانند دستگاه‌های اینترنت اشیا که ده برابر نسل پنجم هستند و تأخیر کم است. همچنین ظهور کاربردهای مبتنی بر هوش مصنوعی نیازمند یکپارچه‌سازی هوش مصنوعی و ارتباطات است 
\cite{ComprehensiveReview, SignalProcessing}.
روش‌های سنتی ارتباطات بی‌سیم، که اغلب بر اصول بهینه‌سازی تکرارشونده و تعامد منابع تکیه دارند، اغلب به دلیل پیچیدگی بالا و زمان اجرای طولانی که اغلب ناشی از متغیر بودن پارامترهای محیطی است برای برآوردن الزامات بی‌سابقه سناریوهای بلادرنگ در شبکه‌های نسل جدید ناکافی هستند 
\cite{ComprehensiveReview, SignalProcessing}. 

بسیاری از مسائل حیاتی، مانند تخصیص منابع، 
\gls*{Nonconvex}
  هستند و یافتن راه‌حل بهینه سراسری برای آن‌ها دشوار است 
\cite{OptimizingWireless, NeuralSumRate, OpenRANet, UnfoldingWMMSE}.
\gls*{Deep Unfolding}
  به عنوان یک رویکرد
\gls*{Deep Learning}
   مدل-محور  مطرح می‌شود 
\cite{ComprehensiveReview}.
 این سازوکار به منظور بهره‌گیری همزمان از مزایای 
\gls*{Domain Knowledge}
  الگوریتم‌های بهینه‌سازی مرسوم و کارایی محاسباتی مدل‌های 
\gls*{Deep Learning}
   توسعه یافته است 
\cite{ComprehensiveReview, SignalProcessing, UnfoldingWMMSE}. 
در این سمینار، ضمن معرفی حوزه پژوهشی
 «\gls*{Deep Unfolding}»
 بررسی می‌شود که چگونه می‌توان راه‌حل‌هایی تفسیرپذیر، مقیاس‌پذیر و بسیار کارآمد برای چالش‌های لایه فیزیکی و تخصیص منابع در شبکه‌های پیچیده و توزیع‌شده نسل آینده ارائه داد 
\cite{ComprehensiveReview, UnfoldingWMMSE}.

\vskip 5mm
\noindent\textbf{کلمات کلیدی}: 
گسترش ژرف، یادگیری ژرف،
\gls*{Machine Learning}،
مدل-محور، شبکه‌های ارتباطی بی‌سیم، پردازش سیگنال، لایه فیزیکی، تخصیص منابع